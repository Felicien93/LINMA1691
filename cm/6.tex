\section{Coloriages d'arêtes}
%il manque encore qq définitions

%dessin d'un graphe avec professeur et classes
\paragraph{Problème des horaires}
Chaque professeur doit enseigner à un certain nombre de classes pendant un certain nombre d'heures. On veut créer
un horaire sur le plus petit nombre de période possible
\\On relie chaque professeur aux classes auxquelles il donne cours en veillant a colorier les arêtes en fonction des tranches horaires. Deux arêtes de la même couleur ne peuvent pas partir du même nœud. 



\begin{mytheo}[König]
  Pour tout graphe biparti: $\chi '= degré max$
  \begin{proof}
    On va utiliser le théorème de Hall pour les graphes bipartis réguliers (qui ont toujours un couplage parfait).
    \begin{enumerate}
    
    
    \item Soit un graphe biparti $k$-régulier. Par le théorème de Hall, il existe un couplage parfait. On le colorie en couleur $c_{1}$. On considère ensuite les arêtes restantes non encore coloriées: elles forment un graphe $k-1$ régulier. On recommence pour la couleur $c_{2}$ avec un autre couplage. On continue jusqu'à épuisement, on obtient alors $\chi '=k$
    \item Pour un graphe biparti quelconque de degré $k$.
    \begin{itemize}
    \item Ajouter des nœuds d'un côté si nécessaire pour avoir le même nombre de nœuds de chaque côté.
    \item Si tous les nœuds ne sont pas de degré $k$, alors il y a au moins 1 nœuds de degré $<k$ de chaque côté. On ajoute alors une arête entre eux. On recommence jusqu'à $k$-régularité.
    \end{itemize}
    Par le point 1. , il existe un coloriage propre à $k$ couleurs. On supprime ensuite les arêtes et nœuds ajoutés: on obtient un coloriage propre pour le graphe de départ.
    $$\Rightarrow deg max \le \chi ' \le k=deg max$$
    $$\Rightarrow \chi ' = k$$
    \end{enumerate}
  \end{proof}
\end{mytheo}
\begin{mytheo} [Vizing]
Pour tout graphe: $\chi ' = deg max$ ou  $\chi ' = deg max + 1$
  \begin{proof} On sait que $\chi' \ge deg max$, il faut donc prouver que $\chi ' \le deg max + 1$.
  \\On le prouve par induction sur le nombre d'arêtes du graphe.
  \\ \textbf{Pas inductif:} Vrai pour $m$ arêtes. Soit un graphe à  $m+1$ arêtes, de degré max $k$. Je retire une de ces arêtes: il existe un coloriage propre à $\le k+1$ couleurs.
  \begin{itemize}
  \item Si $\le k$ couleurs: je choisis (k+1) couleurs pour la $(m+1)^{ième}$ arête.
  \item Si $k+1$ couleurs $c_{1},...,c_{k+1}$: je rétablis la $(m+1)^{ième}$ arête: il faut trouver une couleur pour cette arête.
  \end{itemize}

  \end{proof}
\end{mytheo}
%il faut encore ajouter les différents exemple
\begin{myexem}



\end{myexem}
