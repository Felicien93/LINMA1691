\section{Graphes hamiltoniens}
\index{chemin!chemin hamiltonien}
\begin{mydef}
  Un \emph{chemin} est \emph{hamiltonien} s’il passe par chaque noeud du graphe une et une seule fois.
\end{mydef}

\index{cycle!cycle hamiltonien}
\begin{mydef}
  Un \emph{cycle} est \emph{hamiltonien} s’il passe par chaque noeud du graphe une et une seule fois.
\end{mydef}

\index{graphe!graphe hamiltonien}
\begin{mydef}
  Un \emph{graphe} est \emph{hamiltonien} s’il possède un cycle hamiltonien.
\end{mydef}

\begin{mytheo} [Condition nécessaire pour un graphe hamiltonien] 
  Si on ôte $k$ noeuds quelconques d’un graphe hamiltonien, on obtient au plus $k$ composantes connexes.
  \begin{proof}
     Preuve
  \end{proof}
\end{mytheo}
\begin{myexem}
  Exemple
\end{myexem}

\begin{mytheo} [Condition suffisante pour un graphe hamiltonien] 
  Un graphe simple à $n \geq 3$ noeuds tel que le degré minimum est d’au moins $n/2$ est hamiltonien.
  \begin{proof}
     Preuve
  \end{proof}
\end{mytheo}

\index{problème}
\index{problème!du postier chinois}
\begin{mydef} [Problème du postier chinois]
  Dans un graphe pondéré, trouver le parcours fermé le plus court qui passe par toutes les arêtes au moins une fois.
\end{mydef}

\index{problème!du voyageur de commerce}
\begin{mydef} [Problème du voyageur de commerce]
  Dans un graphe pondéré, trouver le parcours fermé le plus court qui passe par tous les noeuds au moins une fois.
\end{mydef}