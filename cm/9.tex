\section{Graphes planaires}
\subsection{Graphes planaires}
\begin{mytheo} [Fáry]
  Tout graphe planaire simple peut être représenté en n’utilisant que des arêtes droites.
  \begin{proof}
    Preuve \textcolor{red}{TODO}
  \end{proof}
\end{mytheo}

\begin{mytheo}
  Le graphe complet $K_5$ à cinq noeuds n’est pas planaire.
  \begin{proof}
    Preuve \textcolor{red}{TODO}
  \end{proof}
\end{mytheo}

\begin{mytheo}
  Le graphe complet biparti $K_{3,3}$ à $3 + 3$ noeuds n’est pas planaire.
  \begin{proof}
    Preuve \textcolor{red}{TODO}
  \end{proof}
\end{mytheo}

\index{subdivision}
\begin{mydef}
  Une \emph{subdivision} (remplacement de chaque arête par un chemin) d’un graphe non planaire est non-planaire, et un sous-graphe d’un graphe planaire est planaire.
\end{mydef}

\begin{mytheo} [Kuratowski]
  Un graphe est non planaire si et seulement s’il contient comme sous-graphe $K_5$ ou $K_{3,3}$ ou une subdivision de ceux-ci.
  \begin{proof}
    Preuve \textcolor{red}{TODO}
  \end{proof}
\end{mytheo}

\index{face}
\index{face!face extérieure}
\index{face!face intérieure}
\begin{mydef}
  Un graphe planaire (dans une représentation sans croisement) découpe le plan en plusieurs régions connexes (au sens géométrique). Ces régions sont appelées \emph{faces}. Il y a une et une seule face non bornée, nommée \emph{face extérieure}, les autres faces sont \emph{intérieures}.
\end{mydef}

\index{face!bord d'une face}
\index{face!face incidente}
\index{face!degré d'une face}
\begin{mydef}
  On identifie le \emph{bord d’une face} au parcours fermé qui longe la face. Le bord parcourt chaque arête une ou deux fois. 
  Une face est \emph{incidente} aux arêtes et sommets qui sont sur son bord.
  Le \emph{degré d’une face} est la longueur du bord, donc le nombre d’arêtes incidentes (comptées une ou deux fois).
\end{mydef}
\begin{myexem}
  Exemple \textcolor{red}{TODO}
\end{myexem}

\index{dual}
\begin{mydef}
  Etant donné un graphe planaire $G$ (dans une représentation sans croisement), construisons $G^*$ , graphe dont les sommets sont les faces de $G$, reliés si et seulement si les faces correspondantes ont dans $G$ une arête en commun. Ce graphe $G^*$ est le \emph{dual} de $G$ (dans cette représentation).
\end{mydef}

\begin{mytheo}
  La somme des degrés des faces est deux fois le nombre d’arêtes.
  \begin{proof}
    Preuve \textcolor{red}{TODO}
  \end{proof}
\end{mytheo}

\begin{mytheo}
  Un graphe est planaire si et seulement si il est représentable sur la sphère sans croisement d’arêtes.
  \begin{proof}
    Preuve \textcolor{red}{TODO}
  \end{proof}
\end{mytheo}

\subsection{Formule d'Euler}
\begin{mytheo} [Formule d'Euler]
  Dans un graphe planaire connexe à $n$ sommets, $e$ arêtes et $f$ faces:\\
  $n−e+f =2$
  \begin{proof}
    Preuve \textcolor{red}{TODO}
  \end{proof}
\end{mytheo}

\begin{mytheo}
  \label{theo:threensix}
  Dans tout graphe planaire \emph{simple} à $n \geq 3$ sommets et $e$ arêtes,
  $e \leq 3n - 6$.
  \begin{proof}
    Preuve \textcolor{red}{TODO}
  \end{proof}
\end{mytheo}

\begin{mytheo}
  Pour tout graphe planaire \emph{simple}, il y a un noeud de degré $\leq 5$.
  \begin{proof}
    On va montrer que le degré moyen est $< 6$.
    Ce qu'il voudra dire qu'il existe un noeud de degré $\leq 5$.
    \begin{align*}
      \deg_{\mathrm{avg}} & = \frac{\sum_{v\in V} \deg(v)}{|V|}\\
                          & = \frac{2|E|}{|V|}.
    \end{align*}
    Considérons 2 cas
    \begin{itemize}
      \item Si $|V| < 3$, l'énoncé est trivial car dans un graphe simple,
        pour tout $v \in V$, $\deg(v) \leq |V|-1$ du coup
        $\deg(v) \leq |V| - 1 < 2 \leq 5$ pour tout $v$.
      \item
        Comme notre graphe est simple,
        on peut utiliser le théorème~\ref{theo:threensix},
        on a donc $|E| \leq 3|V| - 6$.
        Dès lors
        \begin{align*}
          \deg_{\mathrm{avg}} & \leq 2\frac{3|V|-6}{|V|}\\
                              & = 6 - \frac{12}{|V|} < 6.
        \end{align*}
    \end{itemize}
  \end{proof}
\end{mytheo}

\begin{mycorr}
  $K_5$ est non planaire.
  \begin{proof}
    $K_5$ a 5 noeuds et 10 arêtes.
    Par le théorème~\ref{theo:threensix}, $|E| \leq 3|V| - 6$.
    Il faut donc que $10 \leq 3 \cdot 5 - 6 = 9$, ce qui est faux.
    Le graphe est par conséquent non planaire.
  \end{proof}
\end{mycorr}

\begin{mycorr}
  $K_{3,3}$ est non planaire.
  \begin{proof}
    $K_{3,3}$ a 6 noeuds et 9 arêtes.
    C'est un graphe biparti donc les cycles sont de longueur pair de plus il est simple donc tous les cycles ont une longueur $\geq 4$.
    Donc toutes les faces ont un degré $\geq 4$.
    On a alors $\sum_{f \in F} \deg(f) \geq 4|F|$ et par le théorème des poignées de main dual, $\sum_{f \in F} \deg(f) = 2|E| = 18$.
    Donc $|F| \leq \frac{18}{4} = 4.5$.

    Par la formule d'Euler, il faut que
    $|F| - |E| + |V| = 2$.
    Or $|F| - |E| + |V| \leq 4.5 - 9 + 6 = 1.5$.
    $K_{3,3}$ ne peut donc pas être planaire.
  \end{proof}
\end{mycorr}

\subsection{Les cinq solides platoniciens}
\index{solide platonicien}
\begin{mydef}
  Un \emph{solide platonicien} est un polyèdre convexe régulier.
  C'est-à-dire que toutes les faces, sommets et arêtes sont identiques à une rotation près.
\end{mydef}

\begin{mytheo}
  Il y a 5 solides platoniciens.
  \begin{proof}
    Les polyèdres convexes correspondent à des graphes planaires, via projection.
    Le fait qu'ils soient platoniciens nous dit que chaque noeud est de même degré $p$ et que chaque face est de même degré $q$.
    \begin{center}
      \begin{tabular}{ll}
        La formule d'Euler & $|F| - |E| + |V| = 2$\\
        Poignées de main & $p|V| = 2|E|$\\
        Poignées de main dual & $q|F| = 2|E|$
      \end{tabular}
    \end{center}
    Donc
    \begin{align*}
      \frac{2}{q}|E| - |E| + \frac{2}{p} |E| & = 2\\
      \frac{2}{q} - 1 + \frac{2}{p} & = \frac{2}{|E|} > 0\\
      \frac{1}{q} + \frac{1}{p} & > \frac{1}{2}.
    \end{align*}
    On sait donc que soit $p$, soit $q$ est $< 4$ (ou les deux).
    Or $p \geq 3$ et $q \geq 3$ (par géométrie, graphes planaires simple de dual simple).
    Les possibilités sont
    \begin{center}
      \begin{tabular}{|c|c|c|c|c|c|}
        \hline
        $p$ & $q$ & $|V|$ & $|F|$ & $|E|$ & Polyèdre\\
        \hline
         3  &  3  &   4   &   4   &   6   & Tétraèdre\\
         3  &  4  &   8   &   6   &  12   & Cube\\
         4  &  3  &   6   &   8   &  12   & Octaèdre\\
         3  &  5  &  20   &  12   &  30   & Dodécaèdre\\
         5  &  3  &  12   &  20   &  30   & Icosaèdre\\
        \hline
      \end{tabular}
    \end{center}
  \end{proof}
\end{mytheo}

\begin{mytheo} [Kempe]
  Tout graphe planaire possède un coloriage propre à cinq couleurs.
  ``Toute carte peut être coloriée avec 5 couleurs''.\\
  Nombre chromatique $\chi$ (graphe planaire) $\leq 5$.
  \begin{proof}
    Par récurrence: ``on enlève un noeud, on colorie par hyp. de récurrence, on remet le noeud.''
    On peut supposer le graphe \emph{simple} (car arêtesmultiples n'affectent pas $\chi$).
    Il existe un noeud $u$ de degré $\leq 5$.
    On enlève $u$, on a encore un graphe planaire, on le colorie.
    On rétablit $u$:
    \begin{itemize}
      \item si $\deg(u) < 5$: facile, on utilise une couleur non utilisée par les voisins pour $u$.
      \item Si $\deg(u) = 5$: Si ces 5 voisins utilisent $< 5$ couleurs: facile aussi.
        Si 5 couleurs utilisées $c_1, c_2, c_3, c_4, c_5$.
        Regardons $v_1$ et $v_3$. Si $v_1$ et $v_3$ sont sur des composantes connexes différentes: on échange $c_1$ et $c_3$ sur
        la composante connexe ($c_1-c_3$) de $v_3$, et on colorie $u$ en $c_1$ (sur le graphe des noeuds de couleur $c_1$ et $c_3$.
        Si $v_1$ et $v_3$ sont dans la même composante connexe ($c_1-c_3$):
        Maintenant $v_2$ et $v_4$ sont dans des composantes connexes
        différentes (dans le graphe de couleurs $c_2-c_4$).
        Même raisonnement: échanger $c_2$ et $c_4$ sur composante connexe ($v_2$).
    \end{itemize}
  \end{proof}
\end{mytheo}

\begin{mytheo} [Appel, Haken]
  Tout graphe planaire possède un coloriage propre à quatre couleurs.
  \begin{proof}
    Preuve \textcolor{red}{TODO}
  \end{proof}
\end{mytheo}