\section{Graphes connexes, eulériens et bipartis}
\subsection{Graphes}
\begin{mydef}
  Un \emph{graphe} est un triplet ($V$, $E$, $\varphi$), où :\\
  - $V$ est un ensemble dont les éléments sont appelés sommets ou noeuds; \\
  - $E$ est un ensemble dont les éléments sont appelés arêtes; \\
  - $\varphi$ est une fonction, dîte fonction d'incidence, qui associe à chaque arête un sommet ou une paire de sommets. \\
\end{mydef}

\begin{mydef}
  Deux sommets incidents à la même arête sont dits \emph{adjacents}.
\end{mydef}

\begin{mydef}
  Une arête incidente à un seul sommet est une \emph{boucle}.
\end{mydef}

\begin{mydef}
  Le \emph{degré} d'un sommet est le nombre d'arêtes incidentes à celui-ci.
\end{mydef}

\begin{mydef}
  Un \emph{sous-graphe du graphe} ($V$, $E$, $\varphi$) est un graphe ($V'$, $E'$, $\varphi'$) avec : \\
  - $V' \subseteq V$ ; \\
  - $E' \subseteq E$ ; \\
  - $\varphi'$ est la restriction de $\varphi'$ à $E'$.
\end{mydef}

\subsection{Isomorphisme de Graphes}
\begin{mydef}
  Deux graphes ($V$, $E$, $\varphi$) et ($V'$, $E'$, $\varphi'$) sont dits \emph{isomorphes} s'il existe des bijections $f:V \to V'$ et $g:E \to E'$ telles que :
  \begin{center}
    $\varphi(e) = {u, v}$ ssi $\varphi(g(e)) = {f(u), f(v)}$.
  \end{center}
  Deux graphes sont isomorphes s'il y a une bijection entre les noeuds et les arêtes.
\end{mydef}



\emph{}

\begin{mytheo}
  Enoncé.
  \begin{proof}
    Preuve.
  \end{proof}
\end{mytheo}
