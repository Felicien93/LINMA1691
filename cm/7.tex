\section{Cliques, ensembles indépendants et l'impossible désordre}
\subsection{Ensembles indépendants}
\begin{mytheo} [Théorème de l'amitié]
  Parmi six personnes, on en trouve toujours trois qui se connaissent l’une l’autre, ou trois qui sont étrangers l’un à l’autre.
  \begin{proof}
     Preuve \textcolor{red}{TODO}
  \end{proof}
\end{mytheo}

\index{ensemble indépendant}
\begin{mydef}
  Un \emph{ensemble indépendant} d’un graphe est un ensemble de noeuds deux à deux non adjacents.
\end{mydef}

\index{ensemble indépendant!maximum}
\begin{mydef}
  Un \emph{ensemble indépendant maximum} est un ensemble indépendant dont le nombre de noeuds est maximal.
\end{mydef}

\begin{mytheo}
  Un ensemble de noeuds est indépendant si et seulement si son complémentaire est une couverture de sommets.
  \begin{proof}
     Soit S un ensemble de noeuds indépendant. 
 
     S est un ensemble de noeuds indépendant. \\
     $\Leftrightarrow$ Il n'existe pas d'arête rejoignant 2 noeuds de S. \\
     $\Leftrightarrow$ Toute arête a au moins une extrémité qui n'est pas incluse dans S. \\
     $\Leftrightarrow$ Le complémentaire de S est une couverture de sommets.
  \end{proof}
\end{mytheo}

\begin{mycorr}
  $|$ensemble indépendant maximum$| + |$couverture minimum$| =
|$nombre de noeuds$|$
  \begin{proof}
     Preuve \textcolor{red}{TODO}
  \end{proof}
\end{mycorr}

\subsection{Cliques}
\index{clique}
\begin{mydef}
  Une \emph{clique} d’un graphe est un ensemble de noeuds deux à deux adjacents. Autrement dit, c’est un sous-graphe complet.
\end{mydef}

\index{clique!maximum}
\begin{mydef}
  Une \emph{clique maximum} est une clique dont le nombre de noeuds est maximale.
\end{mydef}

\begin{mytheo}
  Un ensemble est indépendant dans un graphe simple si et seulement s’il est une clique dans le graphe complémentaire.
  \begin{proof}
     Soit S un ensemble indépendant dans un graphe simple G.
     
     S est un ensemble indépendant de G \\
     $\Leftrightarrow$ Deux noeuds quelconques de S sont non-adjacents dans G. \\
     $\Leftrightarrow$ Deux noeuds quelconques de S sont adjacents dans le complémentaire de G. \\
     $\Leftrightarrow$ S est une clique dans le complémentaire de G.\\
  \end{proof}
\end{mytheo}

\begin{mytheo} [Théorème de l'amitié]
  Tout graphe simple à six noeuds contient une clique de trois noeuds ou un ensemble indépendant de trois noeuds.
  \begin{proof}
     Preuve \textcolor{red}{TODO}
  \end{proof}
\end{mytheo}

\begin{mytheo} [Théorème de l'amitié]
  En coloriant, de façon arbitraire, les arêtes du graphe complet à six noeuds en bleu et rouge, on crée un triangle bleu ou un triangle rouge.
  \begin{proof}
     Preuve \textcolor{red}{TODO}
  \end{proof}
\end{mytheo}

\begin{mytheo} [Théorème de Ramsey]
  Soit un graphe complet à $r$ noeuds. On colorie les arêtes en les couleurs $c_1$ , ..., $c_k$ . On cherche la plus petite valeur de $r$ tel que tout coloriage crée une clique à $n_1$ noeuds de couleur $c_1$ , ou une clique à $n_2$ noeuds de couleur $c_2$ , ..., ou une clique à $n_k$ noeuds de couleur $c_k$ . Cette plus petite valeur de $r$, est le nombre de Ramsey $R(n_1 , ..., n_k)$.\\
  $R(n_1 , ..., n_k)$ existe !
  \begin{proof}
     Preuve \textcolor{red}{TODO}
  \end{proof}
\end{mytheo}

\begin{mytheo} [Théorème de Erdös et Szekeres]
  Pour $m, n \geq 2: R(m, n) \leq R(m, n-1) + R(m-1, n)$.
  \begin{proof}
     Prenons un noeud quelconque u dans un graphe complet à $R(m,n-1) + R(m-1,n)$ noeuds avec des arêtes coloriées en bleu ou en rouge. Soit M et N définis tels que 
     $$M = \{ \text{voisins de u reliés par des arêtes bleues}\}$$ 
     $$N = \{ \text{voisins de u reliés par des arêtes rouges}\}.$$
 On a (simplement la somme des noeuds) que 
 $$|M|+|N|+1 = R(m,n-1) + R(m-1,n).$$ 
 
 Donc on a que  
 \begin{equation} \label{cm7:RM}
 |M| \geq R(m-1,n)
\end{equation}    
ou bien 
\begin{equation} \label{cm7:RN}
 |N| \geq R(m,n-1)
\end{equation}  

Si on est dans le cas de figure de l'inégalité \ref{cm7:RM}, il y a deux possibilités. Soit il existe une clique rouge de $n$ noeuds dans $M$ ce qui implique qu'il existe une clique rouge de $n$ noeuds dans le graphe. Soit il existe une clique bleue de $m-1$ noeuds dans $M$ ce qui en incluant $u$ fait une clique bleue de $m$ noeuds dans le graphe.

Si on est dans le cas de figure de l'inégalité \ref{cm7:RN}, idem mutatis mutandis. 
  \end{proof}
\end{mytheo}

\begin{mycorr}
  $R(m, n) \leq (
    \begin{array}{c}
      m+n-2 \\
      m-1
    \end{array})$.
  \begin{proof}
     Preuve \textcolor{red}{TODO}
  \end{proof}
\end{mycorr}

\begin{mytheo}
  $R(n_1, ..., n_k) \leq R(n_1, ..., n_{k-2}, R(n_{k-1}, n_k))$.
  \begin{proof}
  Prenons un graphe complet à $R(n_1, ..., n_{k-2}, R(n_{k-1}, n_k))$ noeuds et leurs arêtes coloriées en $k$ couleurs. Soit $c_i $ la couleur $i$.
  
  Faisons semblant que $c_{k-1}$ et $c_{k}$ sont une seule couleur. Cela implique qu'il n'y a plus que $k-1$ couleurs. Il existe donc une clique à $n_1$ noeuds de couleur $c_1$ ou bien une clique à $n_2$ noeuds de couleur $c_2$ et ainsi de suite jusqu'à la possibilité d'une clique de $R(n_{k-1}, n_k)$ noeuds de couleur $c_{k-1}$ ou $c_{k}$. 
  
  Or par définition de $R(n_{k-1}, n_k)$, cette dernière possibilité revient à dire qu'il existe une clique de $n_{k-1}$ noeuds de couleur $c_{k-1}$ ou une clique de $n_{k}$ noeuds de couleur $c_{k}$.
  \end{proof}
\end{mytheo}

\begin{mytheo} [Théorème de l'amitié]
  $R(3, 3) = 6$
  \begin{proof}
     Preuve \textcolor{red}{TODO}
  \end{proof}
\end{mytheo}

\begin{mytheo} [Théorème de Turán]
  Si un graphe simple a strictement plus de $(1 − \frac{1}{r}) \frac{n^2}{2}$ arêtes, alors il a une clique de $r + 1$ noeuds.
  \begin{proof}
     Preuve \textcolor{red}{TODO}
  \end{proof}
\end{mytheo}

\begin{mytheo} [Théorème de Schur]
  Pour chaque $k$ , il y a un nombre $r_k$ tel que pour toute partition des nombres $1, 2, ..., r_k$ en $k$ classes, une de ces classes contient $x, y , z$ tels que $x + y = z$.
  \begin{proof}
    Je prétends qu'on peut prendre $$r_k = R(3,3,\ldots , 3).$$
    
    Soit le graphe à $r_k$ noeuds numérotés ${1, 2, 3, \ldots ,r_k }$. On colorie les noeuds en $k$ couleurs. On attribue à l'arête $ij$ la couleur de noeud $|i-j|$. Par le choix de $r_k$, il existe un triangle monochrome. 
    
    Ils existent donc $i,j,k$ tel que $x=j-i, y = k-j, z = x+y = k-i$ sont de même couleur.
  \end{proof}
\end{mytheo}

\begin{mytheo} [Théorème de Esther Klein]
  Parmi cinq points arbitraires dans le plan, tels que trois d’entre eux ne sont jamais alignés, on peut toujours en choisir quatre qui déterminent un quadrilatère convexe.
  \begin{proof}
     Preuve \textcolor{red}{TODO}
  \end{proof}
\end{mytheo}

\begin{mytheo} [Théorème de Van der Waerden]
  Pour tout $k , l,$ il existe un nombre $W (k , l)$ tel que les nombres de $1$ à $W (k , l)$, coloriés arbitrairement en $k$ couleurs, contiennent une progression arithmétique monochrome de longueur $l$.
  \begin{proof}
     Preuve \textcolor{red}{TODO}
  \end{proof}
\end{mytheo}







