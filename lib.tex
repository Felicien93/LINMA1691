\documentclass[11pt,a4paper]{article}

% French
\usepackage[utf8x]{inputenc}
\usepackage[frenchb]{babel}
\usepackage[T1]{fontenc}
\usepackage{lmodern}
\usepackage{url}

% Math symbols
\usepackage{amsmath}
\usepackage{amssymb}
\usepackage{amsthm}
\usepackage{subfigure} %Allows to have several figures on the same line.
\usepackage{hyperref} %Allows to make references (\ref{}), pdf links are now clickable.

\usepackage{makeidx} %Allows to create an index
\makeindex

% Theorem and definitions
\theoremstyle{definition}
\newtheorem{mydef}{Définition}
\newtheorem{mynota}[mydef]{Notation}
\newtheorem{myprop}[mydef]{Propriétés}
\newtheorem{myrem}[mydef]{Remarque}
\newtheorem{myform}[mydef]{Formules}
\newtheorem{mycorr}[mydef]{Corrolaire}
\newtheorem{mytheo}[mydef]{Théorème}
\newtheorem{mylem}[mydef]{Lemme}
\newtheorem{myexem}[mydef]{Exemple}
\newtheorem{myalgo}[mydef]{Algorithme}

\newcommand{\bigoh}{\mathcal{O}}

\usepackage{tikz}

\definecolor{mygreen}{rgb}{0,0.6,0}
\definecolor{mygray}{rgb}{0.5,0.5,0.5}
\definecolor{mymauve}{rgb}{0.58,0,0.82}

\tikzstyle{vertex}=[circle,fill=gray!50,minimum size=15pt,inner sep=0pt]
\tikzstyle{visited}=[circle,fill=green!25,minimum size=15pt,inner sep=0pt]
\tikzstyle{unvisited}=[circle,fill=blue!25,minimum size=15pt,inner sep=0pt]

\newcommand{\W}{\ {\color{red} \textbf{!!}} \ }
