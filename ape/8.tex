\section{Séance 8}

\textbf{Coloriage de sommets et polynôme chromatique}

\paragraph{1. } Quels sont les graphes de nombre chromatique égal à 1? et égal à 2?

\paragraph{2. } Trouvez le nombre chromatique du graphe de Pétersen, et du graphe biparti complet $K_{5,3}$.

\paragraph{3. } Vrai ou faux? \\
\begin{enumerate}[(a)]
  \item Un graphe de degré maximum 3 peut être colorié avec 4 couleurs.
  \item Un graphe de degré maximum 4 peut être colorié avec 4 couleurs.
  \item Si $G$ contient $K_n$ comme sous-graphe, alors son nombre chromatique est supérieur ou égal à $n$.
  \item Si $G$ est de nombre chromatique égal à $n$, alors $G$ contient $K_n$ comme sous-graphe.
\end{enumerate}



\paragraph{4. } L'algorithme glouton de coloration associé à un graphe $G$ fonctionne comme suit: on parcourt les sommets $v_1,v_2,…,v_n$ de $G$ dans un ordre fixé arbitrairement. Lorsqu'on rencontre le sommet $v_i$, on lui assigne la plus petite couleur qui n'est pas encore utilisée par un de ses voisins.
\begin{enumerate}[(a)]
  \item Montrez que tout graphe $G$ possède une séquence de sommets pour laquelle l'algorithme glouton utilise un nombre minimum de couleurs.
  \item Construisez pour tout $k \geq 1$ un arbre de degré maximum $k$ et une séquence de ses sommets pour laquelle l'algorithme glouton utilise $k+1$ couleurs.
\end{enumerate}




\paragraph{5. } Pour les deux graphes ci-dessous appliquez l'algorithme de coloration glouton, et trouvez $\chi(G)$.


\begin{figure}[h!]
  \centering
  %\begin{center}
  \subfigure[]{
    \begin{tikzpicture}[-,>=stealth',shorten >=1pt,auto]
      \Vertex[x=0 ,y=0]{1}
      \Vertex[x=1 ,y=1]{2}
      \Vertex[x=2,y=0]{3}
      \Vertex[x=1 ,y=-1]{4}
      \Vertex[x=3 ,y=1]{5}
      \Vertex[x=3 ,y=-1]{6}
      \Vertex[x=4 ,y=0]{7}



      \path[every node/.style={font=\sffamily\small}]
      (1) edge node [left] {} (2)
      edge node [left] {} (5)
      edge node [left] {} (3)
      edge node [left] {} (4)
      edge node [left] {} (6)

      (2) edge node [right] {} (3)
      edge node [right] {} (7)
      edge node [right] {} (5)

      (3) edge node [right] {} (4)
      edge node [left] {} (5)
      edge node [left] {} (6)
      edge node [left] {} (7)

      (4) edge node [right] {} (6)
      edge node [right] {} (7)

      (5) edge node [right] {} (7)

      (6) edge node [right] {} (7);

  \end{tikzpicture} }
  \subfigure[]{
    \begin{tikzpicture}[-,>=stealth',shorten >=1pt,auto]
      \Vertex[x=0 ,y=0]{1}
      \Vertex[x=1 ,y=1]{2}
      \Vertex[x=1,y=0]{3}
      \Vertex[x=1 ,y=-1]{4}
      \Vertex[x=3 ,y=1]{5}
      \Vertex[x=3 ,y=-1]{6}
      \Vertex[x=2 ,y=0]{7}

      \path[every node/.style={font=\sffamily\small}]
      (1) edge node [left] {} (2)
      edge node [left] {} (3)
      edge node [left] {} (4)

      (2) edge node [right] {} (3)
      edge node [right] {} (7)
      edge node [right] {} (5)

      (3) edge node [right] {} (4)
      edge node [left] {} (7)

      (4) edge node [right] {} (6)
      edge node [right] {} (7)

      (5) edge node [right] {} (7)
      edge node [right] {} (6)

      (6) edge node [right] {} (7);

    \end{tikzpicture}
  }

\end{figure}

\paragraph{6. } Sous quelle condition la somme des coefficients d'un polynôme chromatique est-elle nulle?

\paragraph{7. } Trouvez le polynôme chromatique du graphe biparti complet $K_{2,2}$, et donnez le nombre de colorations possibles du cycle $C_4$ avec 5 couleurs.

\paragraph{8. }
\begin{enumerate}[(a)]
  \item Trouvez le polynôme chromatique d'un graphe chemin de longueur $n$
  \item Utilisez ce résultat pour trouver le polynôme chromatique d'un graphe circuit de longueur $n$.
\end{enumerate}

\paragraph{9. } Soit le graphe parallélépipède $3 \times 3 \times 2$.

\begin{enumerate}[(a)]
  \item Chaque arête du graphe est de longueur 1. On souhaite décrire un chemin qui part du sommet $s$ et arrive au sommet $t$ en passant au moins une fois par chaque arête. Quelle est la longueur minimale d'un tel chemin?
  \item Quel est le nombre minimum de couleurs nécessaires pour colorier les sommets de façon à ce que deux sommets adjacents soient toujours de couleurs différentes?
  \item Soit $P_G(k)$ le polynôme chromatique de ce graphe. Quel est le degré de $P_G(k)$? Que vaut $P_G(2)$?
  \item Trouvez une expression aussi explicite que possible pour le nombre de chemins de longueur $k$ du sommet $s$ à lui-même.
\end{enumerate}

\begin{figure}[h!]
  \centering
  %\includegraphics[scale=0.7]{img_tp8.jpg}
\end{figure}
