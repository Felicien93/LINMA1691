\section{Ape 1}

\subsection{Vrai ou Faux}
\begin{enumerate}
	\item{L'élimination d'un sommet de degré maximum peut augmenter le degré moyen d'un graph.}
	\item{Un graph qui ne contient pas de triangle est pibarti}
	\item{Deux graphes qui possèdent un même nombre de sommets et dont les listes de degrés sont identiques sont isomorphes. Même question, si, en plus, les graphes sont connexes.}
	\item{Un graphe simple de 2 sommets au moins possède toujours 2 sommets de degré identiques}
	\item{$\exists$ un graph simple dont les degrés des sommets sont $\{1,2,2,3,3,4\}$}
	\item{$\exists$ un graph simple dont les degrés des sommets sont $\{1,1,1,2,3,4,6\}$}	
\end{enumerate}

\subsection{Démontrez}
\begin{enumerate}
\item{D'un sommet de degré impair, $\exists$ toujours un chemin jusqu'à un autre sommet de degré impair.}
\item{Pour un graph simple et connexe de n sommets, on a $(n-1) \leq |E| \leq \frac{n \times (n-1)}{2}$}
\item{Un graphe simple qui possède plus de $\frac{(n-1) \times (n-2)}{2}$ arêtes est connexe.}
\item{Tout graphe qui possède n sommets et h arêtes possède au moins $n-k$ composantes connexes}
\end{enumerate}

\subsection{Graphe de l'hypercube}
