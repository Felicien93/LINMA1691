\documentclass[11pt,a4paper]{article}

% French
\usepackage[utf8x]{inputenc}
\usepackage[frenchb]{babel}
\usepackage[T1]{fontenc}
\usepackage{lmodern}
\usepackage{url}

% Math symbols
\usepackage{amsmath}
\usepackage{amssymb}
\usepackage{amsthm}
\usepackage{subfigure} %Allows to have several figures on the same line.

% Theorem and definitions
\theoremstyle{definition}
\newtheorem{mydef}{Définition}
\newtheorem{mynota}[mydef]{Notation}
\newtheorem{myprop}[mydef]{Propriétés}
\newtheorem{myrem}[mydef]{Remarque}
\newtheorem{myform}[mydef]{Formules}
\newtheorem{mycorr}[mydef]{Corrolaire}
\newtheorem{mytheo}[mydef]{Théorème}
\newtheorem{mylem}[mydef]{Lemme}
\newtheorem{myexem}[mydef]{Exemple}
\newtheorem{myalgo}[mydef]{Algorithme}

\newcommand{\bigoh}{\mathcal{O}}

\usepackage{tikz}

\definecolor{mygreen}{rgb}{0,0.6,0}
\definecolor{mygray}{rgb}{0.5,0.5,0.5}
\definecolor{mymauve}{rgb}{0.58,0,0.82}

\tikzstyle{vertex}=[circle,fill=gray!50,minimum size=15pt,inner sep=0pt]
\tikzstyle{visited}=[circle,fill=green!25,minimum size=15pt,inner sep=0pt]
\tikzstyle{unvisited}=[circle,fill=blue!25,minimum size=15pt,inner sep=0pt]

\newcommand{\W}{\ {\color{red} \textbf{!!}} \ }


\usepackage[framemethod=tikz]{mdframed}
\usepackage{tikzpagenodes}
\usetikzlibrary{calc}

%http://tex.stackexchange.com/questions/107191/indented-box-that-split-in-multiple-pages
\mdfdefinestyle{mysquare}{%
  leftmargin=0pt,
  rightmargin={\dimexpr4pt+2ex\relax},
  innertopmargin=2\baselineskip,
  skipabove={\dimexpr0.5\baselineskip+\topskip\relax},
  skipbelow={\dimexpr0.5\baselineskip+\topskip\relax},
  singleextra={% Single extra applies when it fits in a single page
  \path let \p1=(P), \p2=(O)
    in node[font=\bfseries] at ([yshift=-2ex]0.5*\x1-\x2,\y1) {Solution};
  \fill[black] ([xshift=2pt,yshift=2pt]P) rectangle ++(1ex,1ex);
  \fill[black] ([xshift=-2pt,yshift=-2pt]O) rectangle ++(-1ex,-1ex);
  \fill[black] ([xshift=-2pt,yshift=2pt]O|-P) rectangle ++(-1ex,1ex);
  \fill[black] ([xshift=2pt,yshift=-2pt]O-|P) rectangle ++(1ex,-1ex);
  },
  firstextra={% First extra applies on the first page when it doesn't fit in one page
  \path let \p1=(P), \p2=(O)
    in node[font=\bfseries] at ([yshift=-2ex]0.5*\x1-\x2,\y1) {Solution};
  \fill[fill=black] ([xshift=2pt,yshift=2pt]P) rectangle ++(1ex,1ex);
  \fill[black] ([xshift=-2pt,yshift=2pt]O|-P) rectangle ++(-1ex,1ex);
  },
  secondextra={% First extra applies on the last page when it doesn't fit in one page
  \fill[fill=black] ([xshift=2pt,yshift=-2pt]O-|P) rectangle ++(1ex,-1ex);
  \fill[black] ([xshift=-2pt,yshift=-2pt]O) rectangle ++(-1ex,-1ex);
  }
}
\newmdenv[style=mysquare]{solution}

\newcommand{\nosolution}
{Cet exercice ne contient pas encore de solution.
Vous êtes invité à nous en soumettre une à l'adresse suivante
\begin{center}
\url{https://github.com/blegat/LINMA1702-exercices}
\end{center}
ou par mail.}

\parindent0mm

\setlength{\textwidth}{15cm}
\setlength{\oddsidemargin}{0.50cm}
\setlength{\textheight}{8.7in}
\setlength{\topmargin}{-0.5in}

\newcommand{\R}{{\mbox{\bf R}}}

\author{}

\title{LINMA1691 -- Théorie et algorithmique des graphes : Exercices}

\begin{document}

\maketitle

\begin{center}
\textbf{Code source et bug tracker}\\
\url{https://github.com/blegat/LINMA1691-exercices}
\end{center}

\tableofcontents

\section{Graphes connexes, eulériens et bipartis}
\subsection{Graphes}

\index{graphe}
\begin{mydef}
  Un \emph{graphe} est un triplet ($V$, $E$, $\varphi$), où :\\
  - $V$ est un ensemble dont les éléments sont appelés sommets ou noeuds; \\
  - $E$ est un ensemble dont les éléments sont appelés arêtes; \\
  - $\varphi$ est une fonction, dîte fonction d'incidence, qui associe à chaque arête un sommet ou une paire de sommets.
\end{mydef}

\index{sommets adjacents}
\begin{mydef}
  Deux sommets incidents à la même arête sont dits \emph{adjacents}.
\end{mydef}

\index{boucle}
\begin{mydef}
  Une arête incidente à un seul sommet est une \emph{boucle}.
\end{mydef}

\index{degré}
\begin{mydef}
  Le \emph{degré} d'un sommet est le nombre d'arêtes incidentes à celui-ci.
\end{mydef}

\index{graphe!sous-graphe}
\begin{mydef}
  Un \emph{sous-graphe du graphe} ($V$, $E$, $\varphi$) est un graphe ($V'$, $E'$, $\varphi'$) avec : \\
  - $V' \subseteq V$ ; \\
  - $E' \subseteq E$ ; \\
  - $\varphi'$ est la restriction de $\varphi'$ à $E'$.
\end{mydef}

\index{isomorphisme}
\subsection{Isomorphisme de Graphes}
\begin{mydef}
  Deux graphes ($V$, $E$, $\varphi$) et ($V'$, $E'$, $\varphi'$) sont dits \emph{isomorphes} s'il existe des bijections $f:V \to V'$ et $g:E \to E'$ telles que :
  \begin{center}
    $\varphi(e) = \{u, v\}$ ssi $\varphi(g(e)) = \{f(u), f(v)\}$.
  \end{center}
  Deux graphes sont isomorphes s'il y a une bijection entre les noeuds et les arêtes.
\end{mydef}

\begin{myexem}
  Voici deux exemples d'isomorphisme du même graphe :
    \begin{figure} [!h]
      \centering  
	    \subfigure[]
    	{  
    	  \begin{tikzpicture}[scale = 0.75]  
          \node[draw, circle] at ( 0, 1.5)  (1) {1};
          \node[draw, circle] at ( 2, 0  )  (2) {2};
          \node[draw, circle] at ( 1,-1.5)  (3) {3};
          \node[draw, circle] at (-1,-1.5)  (4) {4};
          \node[draw, circle] at (-2, 0  )  (5) {5};

          \draw[-] (1) edge [bend left] node[anchor = south] {a} (2);
          \draw[-] (2) edge [bend left] node[anchor = west]  {b} (3);
          \draw[-] (3) edge [bend left] node[anchor = north] {c} (4);
          \draw[-] (4) edge [bend left] node[anchor = east]  {d} (5);
          \draw[-] (5) edge [bend left] node[anchor = east]  {e} (1);
        \end{tikzpicture}
      }
      % Cette ligne de commentaire semble être nécessaire pour que les figures soient affichées sur une ligne
      \subfigure[]  
      {  
        \begin{tikzpicture}[scale = 0.75]
          \node[draw, circle] at ( 0, 1.5)  (1) {$1'$};
          \node[draw, circle] at ( 2, 0  )  (2) {$2'$};
          \node[draw, circle] at ( 1,-1.5)  (3) {$3'$};
          \node[draw, circle] at (-1,-1.5)  (4) {$4'$};
          \node[draw, circle] at (-2, 0  )  (5) {$5'$};

          \draw[-] (1) edge node {$a'$} (3); %TODO mettre le label autrement pour que ce soit lisible
          \draw[-] (2) edge node {$b'$} (5); %TODO mettre le label autrement pour que ce soit lisible
          \draw[-] (2) edge node {$c'$} (4); %TODO mettre le label autrement pour que ce soit lisible
          \draw[-] (3) edge node {$d'$} (5); %TODO mettre le label autrement pour que ce soit lisible
          \draw[-] (1) edge node {$e'$} (4); %TODO mettre le label autrement pour que ce soit lisible

          %TODO tracer les lignes qui montrent comment on passe du graphe b au graphe c (en gardant les 3 graphes sur la même ligne)

        \end{tikzpicture}  
      }
      % Cette ligne de commentaire semble être nécessaire pour que les figures soient affichées sur une ligne
      \subfigure[]  
      {  
        \begin{tikzpicture}[scale = 0.75]
          \node[draw, circle] at ( 0, 1.5)  (1) {$1'$};
          \node[draw, circle] at ( 2, 0  )  (4) {$4'$};
          \node[draw, circle] at ( 1,-1.5)  (2) {$2'$};
          \node[draw, circle] at (-1,-1.5)  (5) {$5'$};
          \node[draw, circle] at (-2, 0  )  (3) {$3'$};

          \draw[-] (1) edge [bend left] node[anchor = south] {$e'$} (4);
          \draw[-] (4) edge [bend left] node[anchor = west]  {$c'$} (2);
          \draw[-] (2) edge [bend left] node[anchor = north] {$b'$} (5);
          \draw[-] (5) edge [bend left] node[anchor = east]  {$d'$} (3);
          \draw[-] (3) edge [bend left] node[anchor = east]  {$a'$} (1);
        \end{tikzpicture}  
      }
    \end{figure}
    Notons que les graphes \emph{b} et \emph{c} sont les mêmes, leurs noeuds ont simplement été réordonnés.\\
    L'isomorphisme entre \emph{a} et les deux autres est donné par : \\
    
    \begin{tabular}{lll}
      $f(1)=1'$ & $g(a)=e'$ \\
      $f(2)=4'$ & $g(b)=c'$ & $\varphi(a) = \{1, 2\}$\\
      $f(3)=2'$ & $g(c)=b'$ & $\varphi'(a') = \{1', 3'\}$\\
      $f(4)=5'$ & $g(d)=d'$ & $\varphi'(g(a)) = \{f(1), f(2)\}$\\
      $f(5)=3'$ & $g(e)=a'$ \\
    \end{tabular}
    \newline
    \newline
    Notons aussi que plusieurs résultats sont possibles : \\
    \begin{figure}[!h]
      \centering  
      \subfigure[]
      {  
        \begin{tikzpicture}[scale = 0.75]  
          \node[draw, circle] at ( 0, 1.5)  (1) {1};
          \node[draw, circle] at ( 2, 0  )  (2) {2};
          \node[draw, circle] at ( 1,-1.5)  (3) {3};
          \node[draw, circle] at (-1,-1.5)  (4) {4};
          \node[draw, circle] at (-2, 0  )  (5) {5};

          \draw[-] (1) edge [bend left] node[anchor = south] {a} (2);
          \draw[-] (2) edge [bend left] node[anchor = west]  {b} (3);
          \draw[-] (3) edge [bend left] node[anchor = north] {c} (4);
          \draw[-] (4) edge [bend left] node[anchor = east]  {d} (5);
          \draw[-] (5) edge [bend left] node[anchor = east]  {e} (1);
        \end{tikzpicture}
      }
      % Cette ligne de commentaire semble être nécessaire pour que les figures soient affichées sur une ligne
      \subfigure[]  
      {  
        \begin{tikzpicture}[scale = 0.75]
          \node[draw, circle] at ( 0, 1.5)  (1) {$1'$};
          \node[draw, circle] at ( 2, 0  )  (2) {$2'$};
          \node[draw, circle] at ( 1,-1.5)  (3) {$3'$};
          \node[draw, circle] at (-1,-1.5)  (4) {$4'$};
          \node[draw, circle] at (-2, 0  )  (5) {$5'$};

          \draw[-] (1) edge node {$a'$} (3); %TODO mettre le label autrement pour que ce soit lisible
          \draw[-] (2) edge node {$b'$} (5); %TODO mettre le label autrement pour que ce soit lisible
          \draw[-] (2) edge node {$c'$} (4); %TODO mettre le label autrement pour que ce soit lisible
          \draw[-] (3) edge node {$d'$} (5); %TODO mettre le label autrement pour que ce soit lisible
          \draw[-] (1) edge node {$e'$} (4); %TODO mettre le label autrement pour que ce soit lisible

          %TODO tracer les lignes qui montrent comment on passe du graphe e au graphe f (en gardant les 3 graphes sur la même ligne)
        \end{tikzpicture}  
      }
      % Cette ligne de commentaire semble être nécessaire pour que les figures soient affichées sur une ligne
      \subfigure[]  
      {  
        \begin{tikzpicture}[scale = 0.75]
          \node[draw, circle] at ( 0, 1.5)  (1) {$1'$};
          \node[draw, circle] at ( 2, 0  )  (3) {$3'$};
          \node[draw, circle] at ( 1,-1.5)  (5) {$5'$};
          \node[draw, circle] at (-1,-1.5)  (2) {$2'$};
          \node[draw, circle] at (-2, 0  )  (4) {$4'$};

          \draw[-] (1) edge [bend left] node[anchor = south] {$a'$} (3);
          \draw[-] (3) edge [bend left] node[anchor = west]  {$d'$} (5);
          \draw[-] (5) edge [bend left] node[anchor = north] {$b'$} (2);
          \draw[-] (2) edge [bend left] node[anchor = east]  {$c'$} (4);
          \draw[-] (4) edge [bend left] node[anchor = east]  {$e'$} (1);
        \end{tikzpicture}  
      }
    \end{figure}
    
    \begin{tabular}{lll}
      $f(1)=1'$ & $g(a)=a'$ \\
      $f(2)=3'$ & $g(b)=b'$ \\
      $f(3)=5'$ & $g(c)=d'$ \\
      $f(4)=2'$ & $g(d)=c'$ \\
      $f(5)=4'$ & $g(e)=e'$ \\
    \end{tabular}
    \newline
    \newline    
    Les six graphes de cet exemple sont isomorphes entre eux.
\end{myexem}

\subsection{Parcours eulérien}
\index{parcours}
\begin{mydef}%TODO définir ce qu'est un parcours
  Un \emph{parcours} est fermé si $v_0 = v_1$.
\end{mydef}

\index{chemin}
\begin{mydef}
  Un \emph{chemin} est un parcours dont tous les sommets sont distincts.
\end{mydef}

\index{cycle}
\begin{mydef}
  Un \emph{cycle} est un parcours fermé dont tous les sommets d'origine et intérieurs sont tous distincts.
\end{mydef}

\index{graphe!graphe connexe}
\begin{mydef}
  Un \emph{graphe} est \emph{connexe} si pour chaque pair de points il existe un parcours qui les relie. Les \emph{composantes connexes} d'un graphe sont ses sous-graphes connexes maximaux.
\end{mydef}

\index{parcours!parcours eulérien}
\index{graphe!graphe eulérien}
\begin{mydef}
  Un \emph{parcours} est \emph{eulérien} s'il visite chaque arête une et une seule fois. Un \emph{graphe} est \emph{eulérien} s'il existe un parcours eulérien fermé.
\end{mydef}

\begin{mytheo} [Théorème d'Euler]
  Un graphe connexe est eulérien ssi tous les sommets sont de degré pair.
  \begin{proof}
     \href{https://dl.dropboxusercontent.com/u/44092863/Graph_Theory_Romain_Capron.pdf}{Voir notes}
  \end{proof}
\end{mytheo}

\begin{mytheo} [Existence d’un parcours eulérien]
  Un graphe connexe possède un parcours eulérien ssi le nombre de noeuds de degré impair est zéro ou deux.
  \begin{proof}
     \href{https://dl.dropboxusercontent.com/u/44092863/Graph_Theory_Romain_Capron.pdf}{Voir notes}
  \end{proof}
\end{mytheo}

\begin{mytheo} [Théorème des poignées de mains]
  La somme des degrés des noeuds d’un graphe est deux fois le nombre d’arêtes.
  \begin{proof}
     \href{https://dl.dropboxusercontent.com/u/44092863/Graph_Theory_Romain_Capron.pdf}{Voir notes}
  \end{proof}
\end{mytheo}

\subsection{Représentation matricielle du graphe}
\index{matrice d'adjacence}
\begin{mydef}
  La \emph{matrice d'adjacence} est une matrice carrée $n$x$n$ dont l'élément $ij$ est le nombre d'arêtes entre les sommets $v_i$ et $v_j$.
\end{mydef}

\index{matrice d'incidence}
\begin{mydef}
  La \emph{matrice d'incidence} est une matrice rectangulaire $n$x$m$ dont l'élément $ij$ est le nombre de fois que le sommet $v_i$ est incident à l'arête $e_j$.
\end{mydef}

\begin{myexem}
  \href{https://dl.dropboxusercontent.com/u/44092863/Graph_Theory_Romain_Capron.pdf}{Voir notes}
\end{myexem}

\begin{mytheo} [Matrice d’adjacence et nombre de parcours]
  Soit $A$ la matrice d'adjacence d'un graphe. Alors l'élément $ij$ de $A^k$ ($k \geq 0$) est le nombre de parcours de longueur $k$ de $v_i$ vers $v_j$.
  \begin{proof}
     \href{https://dl.dropboxusercontent.com/u/44092863/Graph_Theory_Romain_Capron.pdf}{Voir notes}
  \end{proof}
\end{mytheo}

\index{distance entre deux noeuds}
\begin{mydef}
  La \emph{distance $d(u, v)$} entre les noeuds $u$ et $v$ d'un graphe est le nombre d'arêtes minimal d'un parcours entre ces deux noeuds.
\end{mydef}

\begin{mylem}
  Si $u...u'...v'...v$ est un parcours de longeur minimale de $u$ vers $v$, alors le sous-parcours $u'...v'$ est un parcours de longeur minimale de $u'$ vers $v'$.\\
  En particulier, un parcours de longueur minimale est toujours un chemin.
\end{mylem}

\subsection{Graphe biparti}
\index{graphe!graphe biparti}
\begin{mydef}
  Un graphe est \emph{biparti}  s'il existe une partition en deux ensembles $V_1$ et $V_2$ tels que les sommets de $V_1$ ne sont adjacents qu'à des sommets de $V_2$ et vice versa. La bipartition est $(V_1, V_2)$.
\end{mydef}

\begin{myexem}
  \href{https://dl.dropboxusercontent.com/u/44092863/Graph_Theory_Romain_Capron.pdf}{Voir notes}
\end{myexem}

\begin{mytheo} [Graphes bipartis]
  Un graphe est biparti ssi tous ses cycles sont de longueur paire.
  \begin{proof}
     \href{https://dl.dropboxusercontent.com/u/44092863/Graph_Theory_Romain_Capron.pdf}{Voir notes}
  \end{proof}
\end{mytheo}
\section{Les plus courts chemins}
\begin{mydef}
  Une \emph{fonction de poids} sur un graphe ($V$, $E$, $\varphi$) est une fonction $E \to \mathbb{R}$. Un \emph{graphe pondéré} est un graphe muni d’une fonction de poids. Le \emph{poids} ou la \emph{longueur} d’un parcours est la somme des poids des arêtes qui le compose.
\end{mydef}

\begin{mytheo} [Plus court chemin et plus court parcours]
  Pour un graphe avec une fonction de poids $\geq 0$, si le plus court parcours entre $u$ et $v$ est de longueur $d$, alors le plus court chemin entre $u$ et $v$ est aussi de longueur $d$.
  \begin{proof}
     \href{https://dl.dropboxusercontent.com/u/44092863/Graph_Theory_Romain_Capron.pdf}{Voir notes}
  \end{proof}
\end{mytheo}

\subsection{Algorithme de Dijkstra}
\begin{myalgo}[Algorithme de Dijkstra]
\end{myalgo}

\begin{myexem}
  \href{https://dl.dropboxusercontent.com/u/44092863/Graph_Theory_Romain_Capron.pdf}{Voir notes}
\end{myexem}

\begin{mytheo} [L'algorithme de Dijkstra fonctionne]
  Après chaque MISE A JOUR DE $\ell$ dans l’algorithme, les deux propriétés suivantes sont vérifiées :
  \begin{itemize}
    \item pour $v \in S$, $\ell(v) = d(u_0, v)$ et le chemin le plus court de $u_0$ à $v$ reste dans $S$;
    \item pour $v \notin S$, $\ell(v) \geq d(u_0, v)$, et $\ell(v)$ est la longueur du plus court chemin de $u_0$ vers $v$ dont tous les noeuds internes sont dans $S$.
  \end{itemize}
  \begin{proof}
     \href{https://dl.dropboxusercontent.com/u/44092863/Graph_Theory_Romain_Capron.pdf}{Voir notes}
  \end{proof}
\end{mytheo}

\begin{mycorr} [L'algorithme de Dijkstra est correct]
  L’algorithme de Dijkstra est correct.
\end{mycorr}

\begin{mytheo} [L’algorithme de Dijkstra est quadratique]
  L’algorithme de Dijkstra sur un graphe se termine en un temps de l’ordre $n^2$ .
  \begin{proof}
     \href{https://dl.dropboxusercontent.com/u/44092863/Graph_Theory_Romain_Capron.pdf}{Voir notes}
  \end{proof}
\end{mytheo}

\begin{mydef}
  Un \emph{graphe dirigé} est un triplet ($V$, $E$, $\varphi$), où :\\
  - $V$ est un ensemble dont les éléments sont appelés sommets ou noeuds; \\
  - $E$ est un ensemble dont les éléments sont appelés arêtes; \\
  - $\varphi$ est une fonction, dîte fonction d'incidence, qui associe à chaque arête un couple de sommets. \\
\end{mydef}

\begin{myexem}
  \href{https://dl.dropboxusercontent.com/u/44092863/Graph_Theory_Romain_Capron.pdf}{Voir notes}
\end{myexem}

\subsection{Semi-anneaux}
%THIS IS BULLSHIT!
\section{Arbres et connectivité}
\subsection{Arbres}
\index{arbre}
\index{forêt}
\begin{mydef}
  Un \emph{arbre} est un graphe connexe et sans cycle. Une \emph{forêt} est un graphe sans cycle.
\end{mydef}

\index{graphe!sous-graphe sous-tendant}
\index{graphe!sous-graphe couvrant}
\begin{mydef}
  Un \emph{sous-graphe sous-tendant} ou \emph{couvrant} d’un graphe $G$ est un sous-graphe qui contient tous les sommets de $G$.
\end{mydef}

\begin{mytheo} [Arbres sous-tendants]
  Tout graphe connexe contient un arbre sous-tendant.
  \begin{proof}
     \href{https://dl.dropboxusercontent.com/u/44092863/Graph_Theory_Romain_Capron.pdf}{Voir notes} \textcolor{red}{TODO}
  \end{proof}
\end{mytheo}

\begin{mytheo} [Caractérisations des arbres]
  Soit $G$ un graphe à $n$ sommets et $m$ arêtes. Alors les conditions suivantes sont équivalentes :
  \begin{itemize}
    \item $G$ est connexe et sans cycle;
    \item $G$ est sans cycle et $m = n − 1$;
    \item $G$ est connexe et $m = n − 1$;
    \item $G$ est connexe et supprimer une arête quelconque déconnecte $G$;
    \item $G$ est sans cycle et ajouter une arête quelconque crée un et un seul cycle;
    \item Deux noeuds de $G$ sont toujours reliés par un seul chemin.
  \end{itemize}
  La dernière condition implique que G est sans boucle (pour deux noeuds identiques).
  \begin{proof}
     \href{https://dl.dropboxusercontent.com/u/44092863/Graph_Theory_Romain_Capron.pdf}{Voir notes} \textcolor{red}{TODO}
  \end{proof}
\end{mytheo}

\begin{myform} [Formule de Cayley]
  Soit $T(G)$ le nombre d’arbres sous-tendants de $G$, et $e$ une arête quelconque de $G$, qui n’est pas une boucle. \\
  Alors $T(G) = T(G − e) + T (G.e)$.
  \begin{proof}
     \href{https://dl.dropboxusercontent.com/u/44092863/Graph_Theory_Romain_Capron.pdf}{Voir notes} \textcolor{red}{TODO}
  \end{proof}
\end{myform}

\begin{mytheo} [Théorème de Cayley]
  Le nombre d’arbres sous-tendants de $K_n$ est $n^{n−2}$ .
  \begin{proof}
     Preuve \textcolor{red}{TODO}
  \end{proof}
\end{mytheo}

\subsection{Algorithme de Kruskal}
\index{algorithme!algorithme de Kruskal}
\begin{myalgo}[Algorithme de Kruskal]
\end{myalgo}

\begin{myexem}
  \href{https://dl.dropboxusercontent.com/u/44092863/Graph_Theory_Romain_Capron.pdf}{Voir notes} \textcolor{red}{TODO}
\end{myexem}

\begin{mytheo}
  L’algorithme de Kruskal est correct.
  \begin{proof}
     Preuve \textcolor{red}{TODO}
  \end{proof}
\end{mytheo}

\begin{mytheo} [L’algorithme de Kruskal est efficace]
  L’algorithme de Kruskal requiert un temps de calcul de l’ordre de $m.log(m)$ sur un graphe à $m$ arêtes.
  \begin{proof}
     Preuve \textcolor{red}{TODO}
  \end{proof}
\end{mytheo}

\index{coupe de sommets}
\begin{mydef}
  Pour un graphe connexe, une \emph{coupe de sommets} est un ensemble de sommets qui déconnecte le graphe quand on l’en retire.
\end{mydef}

\index{coupe de d'arêtes}
\begin{mydef}
  Pour un graphe connexe, une \emph{coupe de d'arêtes} est un ensemble d’arêtes qui déconnecte le graphe quand on l’en retire.
\end{mydef}

\index{graphe!graphe k-connexe}
\begin{mydef}
  Un graphe est dit \emph{k-connexe} si retirer $k − 1$ noeuds quelconques laisse le graphe connexe. Autrement dit, si toutes les coupes de sommets sont de taille au moins $k$.
\end{mydef}

\index{connectivité}
\begin{mydef}
   La \emph{connectivité} d’un graphe est la taille de la plus petite coupe de sommets. Si tous les $n$ noeuds sont voisins (ex., le graphe complet), la connectivité est définie comme $n − 1$.
\end{mydef}

\index{graphe!graphe k-arête-connexe}
\begin{mydef}
   Un graphe est dit \emph{k-arête-connexe} si retirer $k − 1$ arêtes quelconques laisse le graphe connexe. Autrement dit, si toutes les coupes d’arêtes sont de taille au moins $k$.
\end{mydef}

\index{connectivité!arête-connectivité}
\begin{mydef}
   L’\emph{arête-connectivité} d’un graphe est la taille de la plus petite coupe d’arêtes.
\end{mydef}

\begin{mytheo} [Lien entre les connectivités]
  connectivité $\leq$ arête-connectivité $\leq$ degré minimum. %TODO center this line
  \begin{proof}
     Preuve \textcolor{red}{TODO}
  \end{proof}
\end{mytheo}

\begin{mytheo} [Théorème de Whitney]
  Un graphe à au moins trois noeuds est 2-connexe ssi toute paire de noeuds distincts est reliée par au moins deux chemins dont les noeuds internes sont distincts.
  \begin{proof}
     Preuve \textcolor{red}{TODO}
  \end{proof}
\end{mytheo}

Ce théorème se généralise :

\begin{mytheo} [Théorème de Menger]
  Un graphe à au moins $k + 1$ noeuds est k-connexe ssi toute paire de noeuds distincts est reliée par au moins deux chemins dont les noeuds internes sont distincts.
  \begin{proof}
     Preuve \textcolor{red}{TODO}
  \end{proof}
\end{mytheo}

\begin{mytheo} [Nombre d'arêtes dans un graphe k-connexe]
  Tout graphe k-connexe à $n$ noeuds possède $kn/2$ arêtes au moins.
  \begin{proof}
     Preuve \textcolor{red}{TODO}
  \end{proof}
\end{mytheo}

\begin{mytheo} [Théorème de Harary]
  Le graphe de Harary $H_{k ,n}$ possède $kn/2$ arêtes et est k-connexe.
  \begin{proof}
     Preuve \textcolor{red}{TODO}
  \end{proof}
\end{mytheo}

\begin{myexem}
  Exemples de graphes de Harary. \textcolor{red}{TODO}
\end{myexem}

\section{Graphes hamiltoniens}

\section{Mariages, couplages et couvertures}
\subsection{Couplage}
\index{couplage}
\begin{mydef}
  Un \emph{couplage} dans un graphe est un ensemble $M$ d’arêtes tel que $M$ ne contient pas de boucles et deux arêtes de $M$ n’ont jamais d’extrêmité en commun.
\end{mydef}

\index{couplage!couplage maximum}
\begin{mydef}
  Un \emph{couplage maximum} est un couplage dont le nombre d’arêtes est maximal.
\end{mydef}

\index{couplage!couplage parfait}
\begin{mydef}
  Un \emph{couplage parfait} est un couplage qui est incident à tous les noeuds.
\end{mydef}

\begin{myrem}
  Un couplage parfait, s’il existe, est maximum.
\end{myrem}

\index{chemin!chemin M-alterné}
\begin{mydef}
  Pour un couplage $M$, un \emph{chemin M-alterné} est un chemin qui passe alternativement par une arête de $M$ et par une arête hors de $M$.
\end{mydef}

\index{chemin!chemin M-augmenté}
\begin{mydef}
  Un \emph{chemin M-augmenté} est un chemin M-alterné dont les noeuds d’origine et de destination ne sont pas incident à une arête de $M$.
\end{mydef}

\begin{mytheo} [Berge]
  Un couplage $M$ est maximum si et seulement s’il n’y a pas de chemin M-augmenté.
  \begin{proof}
     Preuve
  \end{proof}
\end{mytheo}
\begin{myexem}
  Exemple
\end{myexem}

\begin{mytheo} [Théorème du mariage ou de Hall]
  Un graphe biparti avec bipartition $(X , Y)$ possède un couplage incident à tous les noeuds de $X$ si et seulement si pour tout ensemble $S \subseteq X$ , le nombre de voisins de $S$ est au moins $|S|$.
  \begin{proof}
     Preuve
  \end{proof}
\end{mytheo}
\begin{myexem}
  Exemple
\end{myexem}

\begin{myrem}
  Un graphe est k-régulier si tous les noeuds sont de degré $k$.
\end{myrem}

\begin{mycorr}
  Tout graphe biparti k-régulier (pour $k > 0$) possède un couplage parfait.
\end{mycorr}

\subsection{Couverture}
\index{couverture de sommets}
\begin{mydef}
  Une \emph{couverture de sommets} d’un graphe est un ensemble de sommets incident à toutes les arêtes.
\end{mydef}

\index{couverture de sommets!minimum}
\begin{mydef}
  Une \emph{couverture de sommets minimum} d’un graphe est une couverture de sommets avec un nombre minimal de sommets.
\end{mydef}

\begin{myrem}
  Si $K$ est une couverture de sommets et $M$ un couplage, alors $|M| \leq |K|$.
\end{myrem}

\begin{myrem}
  Si $K^*$ est une couverture de sommets minimum et $M$ un couplage maximum, alors $|M^*| \leq |K^*|$.
\end{myrem}

\begin{mylem}
  Si $K$ est une couverture de sommets, $M$ un couplage et que $|M| = |K|$, alors $K$ est minimum et $M$ est maximum.
  \begin{proof}
     Preuve
  \end{proof}
\end{mylem}

\begin{mytheo} [König]
  Dans un graphe biparti, si $K^*$ est une couverture de sommets minimum et $M^*$ un couplage maximum, alors $|M^*| = |K^*|$.
  \begin{proof}
     Preuve
  \end{proof}
\end{mytheo}

\subsection{L'algorithme hongrois}
\index{algorithme!algorithme hongrois}
\begin{myalgo}[Algorithme hongrois]
\end{myalgo}
\begin{myexem}
  Exemple
\end{myexem}
































\section{Coloriages d'arêtes}
\subsection{Coloriages d'arêtes}
%il manque encore qq définitions

%dessin d'un graphe avec professeur et classes
\paragraph{Problème des horaires}
Chaque professeur doit enseigner à un certain nombre de classes pendant un certain nombre d'heures. On veut créer
un horaire sur le plus petit nombre de période possible
\\On relie chaque professeur aux classes auxquelles il donne cours en veillant a colorier les arêtes en fonction des tranches horaires. Deux arêtes de la même couleur ne peuvent pas partir du même nœud.

\index{coloriage}
\index{coloriage!coloriage d'arêtes}
\index{coloriage!coloriage d'arêtes propre}
\begin{mydef}
  Un \emph{coloriage} des arêtes d’un graphe en k couleurs est l’assignation à chaque arête d’une couleur 1; 2; ..., ou $k$. Ce coloriage est dit \emph{propre} si deux arêtes adjacentes sont toujours de couleurs différentes.
\end{mydef}

\index{chromatique!indice chromatique}
\begin{mydef}
  L’\emph{indice chromatique} d’un graphe $G$, noté $\chi '$ ($G$) est le nombre minimal de couleurs nécessaire pour obtenir un coloriage propre des arêtes de $G$.
\end{mydef}


\begin{mytheo}[König]
  Pour tout graphe biparti: $\chi '= degré max$
  \begin{proof}
    On va utiliser le théorème de Hall pour les graphes bipartis réguliers (qui ont toujours un couplage parfait).
    \begin{enumerate}


    \item Soit un graphe biparti $k$-régulier. Par le théorème de Hall, il existe un couplage parfait. On le colorie en couleur $c_{1}$. On considère ensuite les arêtes restantes non encore coloriées: elles forment un graphe $k-1$ régulier. On recommence pour la couleur $c_{2}$ avec un autre couplage. On continue jusqu'à épuisement, on obtient alors $\chi '=k$
    \item Pour un graphe biparti quelconque de degré $k$.
    \begin{itemize}
    \item Ajouter des nœuds d'un côté si nécessaire pour avoir le même nombre de nœuds de chaque côté.
    \item Si tous les nœuds ne sont pas de degré $k$, alors il y a au moins 1 nœuds de degré $<k$ de chaque côté. On ajoute alors une arête entre eux. On recommence jusqu'à $k$-régularité.
    \end{itemize}
    Par le point 1. , il existe un coloriage propre à $k$ couleurs. On supprime ensuite les arêtes et nœuds ajoutés: on obtient un coloriage propre pour le graphe de départ.
    $$\Rightarrow deg max \le \chi ' \le k=deg max$$
    $$\Rightarrow \chi ' = k$$
    \end{enumerate}
  \end{proof}
\end{mytheo}

\begin{mytheo} [Vizing]
Pour tout graphe simple: $\chi ' = deg max$ ou  $\chi ' = deg max + 1$
  \begin{proof} On sait que $\chi' \ge deg max$, il faut donc prouver que $\chi ' \le deg max + 1$.
  \\On le prouve par induction sur le nombre d'arêtes du graphe.
  \\ \textbf{Pas inductif:} Vrai pour $m$ arêtes. Soit un graphe à  $m+1$ arêtes, de degré max $k$. Je retire une de ces arêtes: il existe un coloriage propre à $\le k+1$ couleurs.
  \begin{itemize}
  \item Si $\le k$ couleurs: je choisis (k+1) couleurs pour la $(m+1)^{ième}$ arête.
  \item Si $k+1$ couleurs $c_{1},...,c_{k+1}$: je rétablis la $(m+1)^{ième}$ arête: il faut trouver une couleur pour cette arête.
  \end{itemize}

  \end{proof}
\end{mytheo}
%il faut encore ajouter les différents exemple
\begin{myexem}
  Exemples \textcolor{red}{TODO}
\end{myexem}

\section{Cliques, ensembles indépendants et l'impossible désordre}
\subsection{Ensembles indépendants}
\begin{mytheo} [Théorème de l'amitié]
  Parmi six personnes, on en trouve toujours trois qui se connaissent l’une l’autre, ou trois qui sont étrangers l’un à l’autre.
  \begin{proof}
     Preuve \textcolor{red}{TODO}
  \end{proof}
\end{mytheo}

\index{ensemble indépendant}
\begin{mydef}
  Un \emph{ensemble indépendant} d’un graphe est un ensemble de noeuds deux à deux non adjacents.
\end{mydef}

\index{ensemble indépendant!maximum}
\begin{mydef}
  Un \emph{ensemble indépendant maximum} est un ensemble indépendant dont le nombre de noeuds est maximal.
\end{mydef}

\begin{mytheo}
  Un ensemble de noeuds est indépendant si et seulement si son complémentaire est une couverture de sommets.
  \begin{proof}
     Soit S un ensemble de noeuds indépendant. 
 
     S est un ensemble de noeuds indépendant. \\
     $\Leftrightarrow$ Il n'existe pas d'arête rejoignant 2 noeuds de S. \\
     $\Leftrightarrow$ Toute arête a au moins une extrémité qui n'est pas incluse dans S. \\
     $\Leftrightarrow$ Le complémentaire de S est une couverture de sommets.
  \end{proof}
\end{mytheo}

\begin{mycorr}
  $|$ensemble indépendant maximum$| + |$couverture minimum$| =
|$nombre de noeuds$|$
  \begin{proof}
     Preuve \textcolor{red}{TODO}
  \end{proof}
\end{mycorr}

\subsection{Cliques}
\index{clique}
\begin{mydef}
  Une \emph{clique} d’un graphe est un ensemble de noeuds deux à deux adjacents. Autrement dit, c’est un sous-graphe complet.
\end{mydef}

\index{clique!maximum}
\begin{mydef}
  Une \emph{clique maximum} est une clique dont le nombre de noeuds est maximale.
\end{mydef}

\begin{mytheo}
  Un ensemble est indépendant dans un graphe simple si et seulement s’il est une clique dans le graphe complémentaire.
  \begin{proof}
     Soit S un ensemble indépendant dans un graphe simple G.
     
     S est un ensemble indépendant de G \\
     $\Leftrightarrow$ Deux noeuds quelconques de S sont non-adjacents dans G. \\
     $\Leftrightarrow$ Deux noeuds quelconques de S sont adjacents dans le complémentaire de G. \\
     $\Leftrightarrow$ S est une clique dans le complémentaire de G.\\
  \end{proof}
\end{mytheo}

\begin{mytheo} [Théorème de l'amitié]
  Tout graphe simple à six noeuds contient une clique de trois noeuds ou un ensemble indépendant de trois noeuds.
  \begin{proof}
     Preuve \textcolor{red}{TODO}
  \end{proof}
\end{mytheo}

\begin{mytheo} [Théorème de l'amitié]
  En coloriant, de façon arbitraire, les arêtes du graphe complet à six noeuds en bleu et rouge, on crée un triangle bleu ou un triangle rouge.
  \begin{proof}
     Preuve \textcolor{red}{TODO}
  \end{proof}
\end{mytheo}

\begin{mytheo} [Théorème de Ramsey]
  Soit un graphe complet à $r$ noeuds. On colorie les arêtes en les couleurs $c_1$ , ..., $c_k$ . On cherche la plus petite valeur de $r$ tel que tout coloriage crée une clique à $n_1$ noeuds de couleur $c_1$ , ou une clique à $n_2$ noeuds de couleur $c_2$ , ..., ou une clique à $n_k$ noeuds de couleur $c_k$ . Cette plus petite valeur de $r$, est le nombre de Ramsey $R(n_1 , ..., n_k)$.\\
  $R(n_1 , ..., n_k)$ existe !
  \begin{proof}
     Preuve \textcolor{red}{TODO}
  \end{proof}
\end{mytheo}

\begin{mytheo} [Théorème de Erdös et Szekeres]
  Pour $m, n \geq 2: R(m, n) \leq R(m, n-1) + R(m-1, n)$.
  \begin{proof}
     Prenons un noeud quelconque u dans un graphe complet à $R(m,n-1) + R(m-1,n)$ noeuds avec des arêtes coloriées en bleu ou en rouge. Soit M et N définis tels que 
     $$M = \{ \text{voisins de u reliés par des arêtes bleues}\}$$ 
     $$N = \{ \text{voisins de u reliés par des arêtes rouges}\}.$$
 On a (simplement la somme des noeuds) que 
 $$|M|+|N|+1 = R(m,n-1) + R(m-1,n).$$ 
 
 Donc on a que  
 \begin{equation} \label{cm7:RM}
 |M| \geq R(m-1,n)
\end{equation}    
ou bien 
\begin{equation} \label{cm7:RN}
 |N| \geq R(m,n-1)
\end{equation}  

Si on est dans le cas de figure de l'inégalité \ref{cm7:RM}, il y a deux possibilités. Soit il existe une clique rouge de $n$ noeuds dans $M$ ce qui implique qu'il existe une clique rouge de $n$ noeuds dans le graphe. Soit il existe une clique bleue de $m-1$ noeuds dans $M$ ce qui en incluant $u$ fait une clique bleue de $m$ noeuds dans le graphe.

Si on est dans le cas de figure de l'inégalité \ref{cm7:RN}, idem mutatis mutandis. 
  \end{proof}
\end{mytheo}

\begin{mycorr}
  $R(m, n) \leq (
    \begin{array}{c}
      m+n-2 \\
      m-1
    \end{array})$.
  \begin{proof}
     Preuve \textcolor{red}{TODO}
  \end{proof}
\end{mycorr}

\begin{mytheo}
  $R(n_1, ..., n_k) \leq R(n_1, ..., n_{k-2}, R(n_{k-1}, n_k))$.
  \begin{proof}
  Prenons un graphe complet à $R(n_1, ..., n_{k-2}, R(n_{k-1}, n_k))$ noeuds et leurs arêtes coloriées en $k$ couleurs. Soit $c_i $ la couleur $i$.
  
  Faisons semblant que $c_{k-1}$ et $c_{k}$ sont une seule couleur. Cela implique qu'il n'y a plus que $k-1$ couleurs. Il existe donc une clique à $n_1$ noeuds de couleur $c_1$ ou bien une clique à $n_2$ noeuds de couleur $c_2$ et ainsi de suite jusqu'à la possibilité d'une clique de $R(n_{k-1}, n_k)$ noeuds de couleur $c_{k-1}$ ou $c_{k}$. 
  
  Or par définition de $R(n_{k-1}, n_k)$, cette dernière possibilité revient à dire qu'il existe une clique de $n_{k-1}$ noeuds de couleur $c_{k-1}$ ou une clique de $n_{k}$ noeuds de couleur $c_{k}$.
  \end{proof}
\end{mytheo}

\begin{mytheo} [Théorème de l'amitié]
  $R(3, 3) = 6$
  \begin{proof}
     Preuve \textcolor{red}{TODO}
  \end{proof}
\end{mytheo}

\begin{mytheo} [Théorème de Turán]
  Si un graphe simple a strictement plus de $(1 − \frac{1}{r}) \frac{n^2}{2}$ arêtes, alors il a une clique de $r + 1$ noeuds.
  \begin{proof}
     Preuve \textcolor{red}{TODO}
  \end{proof}
\end{mytheo}

\begin{mytheo} [Théorème de Schur]
  Pour chaque $k$ , il y a un nombre $r_k$ tel que pour toute partition des nombres $1, 2, ..., r_k$ en $k$ classes, une de ces classes contient $x, y , z$ tels que $x + y = z$.
  \begin{proof}
    Je prétends qu'on peut prendre $$r_k = R(3,3,\ldots , 3).$$
    
    Soit le graphe à $r_k$ noeuds numérotés ${1, 2, 3, \ldots ,r_k }$. On colorie les noeuds en $k$ couleurs. On attribue à l'arête $ij$ la couleur de noeud $|i-j|$. Par le choix de $r_k$, il existe un triangle monochrome. 
    
    Ils existent donc $i,j,k$ tel que $x=j-i, y = k-j, z = x+y = k-i$ sont de même couleur.
  \end{proof}
\end{mytheo}

\begin{mytheo} [Théorème de Esther Klein]
  Parmi cinq points arbitraires dans le plan, tels que trois d’entre eux ne sont jamais alignés, on peut toujours en choisir quatre qui déterminent un quadrilatère convexe.
  \begin{proof}
     Preuve \textcolor{red}{TODO}
  \end{proof}
\end{mytheo}

\begin{mytheo} [Théorème de Van der Waerden]
  Pour tout $k , l,$ il existe un nombre $W (k , l)$ tel que les nombres de $1$ à $W (k , l)$, coloriés arbitrairement en $k$ couleurs, contiennent une progression arithmétique monochrome de longueur $l$.
  \begin{proof}
     Preuve \textcolor{red}{TODO}
  \end{proof}
\end{mytheo}








\section{Coloriages de sommets}
\subsection{Coloriages de sommets}
\index{coloriage!coloriage de noeuds}
\index{coloriage!coloriage de noeuds propre}
\index{coloriage!coloriage minimum}
\begin{mydef}
  Un \emph{coloriage de noeuds} est l’attribution à chaque noeud d’une couleur.
  Le coloriage est dit \emph{propre} si des noeuds adjacents ont des couleurs différentes. 
  Un coloriage \emph{minimum} est un coloriage propre qui emploie un nombre minimal de couleurs.
\end{mydef}

\subsection{Nombre chromatique, cliques et degrés}
\index{chromatique!nombre chromatique}
\begin{mydef}
  Le \emph{nombre chromatique}, noté $\chi$, d’un graphe est le nombre minimal de couleurs d’un coloriage propre de sommets du graphe.
\end{mydef}

\begin{mytheo}
  $|$clique max$| \leq \chi$
\end{mytheo}

\begin{mytheo}
  $\chi \leq $ degré max +1
  \begin{proof}
    Par induction sur le nombre de noeuds $n$. 
    Trivial pour $n = 1$.
    Soit $G$ graphe de degré max $d$ de $n$ noeuds. On enlève un noeuds $x$. $G-\{x\}$ est un graphe de $n-1$ noeuds de degré max $\geq d$.
    
    Par hypothère d'induction, il existe un coloriage propre de $G-\{x\}$.
    On rétablis $x$ et ses arêtes. $x$ a au plus $d $ voisins.
    
    \begin{itemize}
    \item Si  $G-\{x\}$ a $\leq d$ couleurs, alors ont peut utiliser une $d$ème couleur pour $x$.
    \item  Si $G-\{x\}$ a $\leq d+1$ couleurs, alors $x$ a au plus $d$ voisins. Il existe donc une couleur non-utilisée, libre pour $x$.
    \end{itemize}
  \end{proof}
\end{mytheo}
\subsection{Polynôme chromatique}
Pour un graphe $G$, le nombre de coloriage propres à $k$ couleurs est noté $\pi_k(G)$.
\begin{mytheo}
  Pour toute arête $e$ du graphe $G$, $\pi_k(G) = \pi_k(G-e) - \pi_k(G.e)$ où $G.e$ est le graphe obtenu en contracant l'arête $e$.
  \begin{proof}
  Soit $u$ et $v$ les deux extrémités de $e$.
	\begin{itemize}
	\item $\pi_k(G-e)$ est le nombre de coloriages des noeuds de $G$ à moins de $k$ couleurs qui sont propres sauf possiblement entre $u$ et $v$. On peut avoir $u$ et $v$ de même couleur.
	\item $\pi_k(G.e)$ est le nombre de coloriages des noeuds de $G$ à moins de $k$ couleurs qui sont propres sauf entre $u$ et $v$. On peut avoir $u$ et $v$ de même couleur.
	\end{itemize}
	On a donc finalement $\pi_k(G-e) - \pi_k(G.e)$ est le nombre de coloriages des noeuds de $G$ à moins de $k$ couleurs qui sont propres même entre $u$ et $v$ : on doit avoir $u$ et $v$ de couleurs différentes.
  \end{proof}
\end{mytheo}

\begin{mycorr} [Birkhoff]
  Pour un graphe $G$ à $n$ noeuds, $\pi_k (G)$ est un polynôme monique de degré $n$, de terme constant nul et dont les coefficients alternent en signe.
  \begin{proof}
    Preuve \textcolor{red}{TODO}
  \end{proof}
\end{mycorr}

\end{document}
